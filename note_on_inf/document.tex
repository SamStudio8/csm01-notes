\documentclass[a4paper]{article}

% Margins to 2cm
\usepackage[left=2cm,right=2cm,top=3cm,bottom=2cm]{geometry}

% No identation on paragraphs
\usepackage{parskip}

% Allow for code samples and change captions for "Listings" to "Code Snippet"
\usepackage{minted}
\usepackage{listings}
\renewcommand\listingscaption{Code Snippet}

% URLS
\usepackage{hyperref}

% Prevent hypenation (not the nicest way to do this)
\hyphenpenalty 1000

% Asterism
% http://tex.stackexchange.com/questions/194898/how-to-use-an-asterism-to-separate-sections
\newcommand{\asterism}{\bigskip\par\centerline{*\,*\,*}\medskip\par}%


\begin{document}

\begin{center}
{\huge{Sensible defaults, infinity and None - a note on Practical 3}}
\noindent\makebox[\linewidth]{\rule{\paperwidth}{0.4pt}}
\end{center}

\vskip 0.5cm

Cast your mind back to the final question of Practical 3 where
given a dictionary featuring island names as keys
and their corresponding co-ordinates as lists, you were asked to find the smallest
island. A suggested solution can be seen below in code snippet \ref{list:firstsol}.

\begin{listing}[H]
    \caption[]{A possible solution for Question 8 of Practical 3}
    \label{list:firstsol}
    \begin{minted}[mathescape,
                %linenos,
                numbersep=5pt,
                gobble=8,
                frame=lines,
                framesep=2mm]{python}


        def smallest_island(islands_dict):

            # Set the minimum area found to 1,000,000 and the smallest island name to an empty string
            min_area = 1000000
            smallest = ""

            # Check each island in the dictionary of islands
            # Remember `for key in dictionary' is a shortcut for `for key in dictionary.keys()'
            for island in islands_dict:

                # Get the co-ordinates of the current island from the dictionary
                # and calculate the size with your landRectangleArea function
                coords = islands_dict[island]
                size = landRectangleArea(coords[0], coords[1], coords[2], coords[3])

                # If the current island size is smaller than the minimum area seen so far,
                # update min_area with the new minimum and update smallest to the current island name
                if size < min_area:
                    min_area = size
                    smallest = island

            # Once the loop is done, return the name of the smallest island
            return smallest

    \end{minted}
\end{listing}

Whilst this works, there was some debate as to a sensible default value for the
variables \texttt{min\_area} \footnote{The example answer given on BlackBoard and
in class actually used the variable name \texttt{min}, however this is a
built-in function in Python for returning the smallest element in a list. Python
is happy to let you `overwrite' these functions without warning and care should be
taken to avoid doing so. This is the same reason you shouldn't name your
dictionaries \texttt{dict} or lists \texttt{list}. A list of built-in functions can
be found at {\href{https://docs.python.org/2/library/functions.html}{https://docs.python.org/2/library/functions.html}}.}
and \texttt{smallest}. Hard-coding the minimum value seems
a little arbitrary and could potentially introduce problems later on. What if
one day your code was used to process much larger islands? If all islands had
an area larger than 1,000,000 - the smallest island could not be identified (see
the example in snippet \ref{list:pitfall}).

\begin{listing}[H]
    \caption[]{An example of the potential pitfall introduced by hard-coding \texttt{min\_area}}
    \label{list:pitfall}
    \begin{minted}[mathescape,
                %linenos,
                numbersep=5pt,
                gobble=8,
                frame=lines,
                framesep=2mm]{python}

        >>> min_area = 1000000
        >>> areas = [1000001, 2000000, 2500000, 3000000, 9999999]
        >>> for area in areas:
        ...     if area < min_area:
        ...         min_area = area
        ...
        >>> min_area
        1000000 # Still 1,000,000 despite the smallest area in the list being 1,000,001

        >>> min(areas)
        1000001 # Built-in min function shows 1,000,001 as smallest element of areas list

    \end{minted}
\end{listing}

Note that \texttt{min\_area} is never updated to a new value as none of the entries in
the \texttt{areas} list are smaller than our hard-coded default of 1,000,000.

We can't guarantee we could ever select an appropriate minimum value in this fashion,
even if we chose a higher value as a default. What if somebody tries
to use your program to calculate the smallest island given areas in small units
like $mm\textsuperscript{2}$? Perhaps somebody else would try and use the code
to handle island populations rather than area for very densely populated islands?

A suggestion from the class was to instead hard-code \texttt{min\_area} to be the
largest integer your computer could hold in memory. For most relatively modern desktops
and laptops the largest \texttt{int} Python can store is 9,223,372,036,854,775,807.
You can check this yourself as shown in snippet \ref{list:maxsize} using the \texttt{sys} package\footnote{{\href{https://docs.python.org/2/library/sys.html}{https://docs.python.org/2/library/sys.html}}};
which you may recall having to import for writing to \texttt{stdout} and \texttt{stdin}.
Along with that functionality, the package contains functions and variables that
relate to your system - here we are interested in \texttt{maxsize}.

However, for reasons beyond the scope of this course it should be noted that
Python is actually capable of storing numbers even larger than this and is only limited
by the amount of memory available on your computer.

\begin{listing}[H]
    \caption[]{Finding the largest positive integer supported by Python on your computer.}
    \label{list:maxsize}
    \begin{minted}[mathescape,
                %linenos,
                numbersep=5pt,
                gobble=8,
                frame=lines,
                framesep=2mm]{python}

        >>> import sys
        >>> sys.maxsize
        9223372036854775807
    \end{minted}
\end{listing}

So the definition of ``biggest number" is somewhat muddy here. Either way this feels
almost as arbitrary as our hard-coded million, if not a little less open to the
pitfall discussed. Surely there must be a better way?

You should by now be familiar with Python's built-in \texttt{int} and \texttt{float}
functions, which you will
have been using for converting strings from CSV files and user input to their
numerical whole-number or decimal representations respectively. \texttt{float}
has a special use case for creating floats that represent positive or negative infinity.

Snippet \ref{list:infinity} replaces our hard-coded \texttt{min\_area} with positive infinity.

\begin{listing}[H]
    \caption[]{Setting \texttt{min\_area} to positive infinity.}
    \label{list:infinity}
    \begin{minted}[mathescape,
                %linenos,
                numbersep=5pt,
                gobble=8,
                frame=lines,
                framesep=2mm]{python}


        def smallest_island(islands_dict):

            # Set the minimum area found to positive infinity and smallest island name to empty string
            # Any int or float will always be less than positive infinity
            min_area = float("inf")
            smallest = ""

            # Check each island in the dictionary of islands
            # Remember `for key in dictionary' is a shortcut for `for key in dictionary.keys()'
            for island in islands_dict:

                # Get the co-ordinates of the current island from the dictionary
                # and calculate the size with your landRectangleArea function
                coords = islands_dict[island]
                size = landRectangleArea(coords[0], coords[1], coords[2], coords[3])

                # If the current island size is smaller than the minimum area seen so far,
                # update min_area with the new minimum and update smallest to the current island name
                if size < min_area:
                    min_area = size
                    smallest = island

            # Once the loop is done, return the name of the smallest island
            return smallest

    \end{minted}
\end{listing}

As any \texttt{int} or \texttt{float} will always be smaller than \texttt{inf},
here we can guarantee that whatever the input, there will never be an island area greater than
our initial value for \texttt{min\_area}, without the risk of setting the value to
a large arbitrary number or the faff of querying the computer for the largest number it can
hold in memory.

Similarly if we instead wanted to find the largest island, we could set a \texttt{max\_area}
variable to an initial value of negative infinity with \texttt{float("-inf")}. Any \texttt{int}
or \texttt{float} will be greater than negative infinity.

For completeness, \texttt{float("infinity")} and \texttt{float("-infinity")} are also valid
in both Python 2 and 3 and are semantically equal to \texttt{float("inf")} and \texttt{float("-inf")} but
take longer to type.

\vskip -0.1cm
\asterism
\vskip -0.3cm

Python also has a keyword called \texttt{None}, a special constant that represents
the absence of value. Snippet \ref{list:none} shows how we might use \texttt{None}
in place of our hard-coded value.

\begin{listing}[H]
    \caption[]{Setting \texttt{min\_area} to the \texttt{None} constant.}
    \label{list:none}
    \begin{minted}[mathescape,
                %linenos,
                numbersep=5pt,
                gobble=8,
                frame=lines,
                framesep=2mm]{python}


        def smallest_island(islands_dict):

            # Set the minimum area found to None and the smallest island name to an empty string
            min_area = None
            smallest = ""

            # Check each island in the dictionary of islands
            # Remember `for key in dictionary' is a shortcut for `for key in dictionary.keys()'
            for island in islands_dict:

                # Get the co-ordinates of the current island from the dictionary
                # and calculate the size with your landRectangleArea function
                coords = islands_dict[island]
                size = landRectangleArea(coords[0], coords[1], coords[2], coords[3])

                # Check whether min_area has a value set (it won't if this is the first island)
                if min_area == None:
                    # If min_area is None, set it to be the current island
                    min_area = size
                    smallest = island
                elif size < min_area:
                    # Otherwise, if min_area has a value, check whether the current
                    # island is smaller than the minimum area seen so far and if so,
                    # update min_area with the new minimum and update smallest to the
                    # current island name
                    min_area = size
                    smallest = island

            # Once the loop is done, return the name of the smallest island
            return smallest

    \end{minted}
\end{listing}

Note here that our \texttt{if} statement has changed to first check whether \texttt{min\_area}
is \texttt{None}, before trying to compare the current island's \texttt{size} to it.
In Python 2, any variable: be it a \texttt{str}ing, \texttt{int}, \texttt{float} or even
an empty \texttt{list} or \texttt{dict}, will be greater than \texttt{None}. Which means
if we failed to set \texttt{min\_area} to something other than \texttt{None} before beginning
our comparisons, we'd still end up with a \texttt{min\_area} of \texttt{None} at the end of the loop.
\\Yet, in Python 3, comparisons with \texttt{None} are a little more complicated and can
return an error if the two types used in a comparison do not have a meaningful ordering
\footnote{Those interested can read a little more about this via {\href{https://docs.python.org/3/whatsnew/3.0.html\#ordering-comparisons}{https://docs.python.org/3/whatsnew/3.0.html\#ordering-comparisons}}.}.

As \texttt{min\_area} is \texttt{None} initially, when the code processes its first island
the \texttt{min\_area == None} test will be \texttt{True} and the size of the first island
will become the new \texttt{min\_area} and \texttt{smallest} will be assigned its name.

All islands processed after the first will use the \texttt{elif} part of the statement
(as \texttt{min\_area == None} will now always be \texttt{False}) and work as before
to compare the current island \texttt{size} to the \texttt{min\_area} seen so far.

\end{document}
