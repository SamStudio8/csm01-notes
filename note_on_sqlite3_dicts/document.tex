\documentclass[a4paper]{article}

% Margins to 2cm
\usepackage[left=2cm,right=2cm,top=3cm,bottom=2cm]{geometry}

% No identation on paragraphs
\usepackage{parskip}

% Allow for code samples and change captions for "Listings" to "Code Snippet"
\usepackage{minted}
\usepackage{listings}
\renewcommand\listingscaption{Code Snippet}

% URLS
\usepackage{hyperref}

% Prevent hypenation (not the nicest way to do this)
\hyphenpenalty 1000

% Asterism
% http://tex.stackexchange.com/questions/194898/how-to-use-an-asterism-to-separate-sections
\newcommand{\asterism}{\bigskip\par\centerline{*\,*\,*}\medskip\par}%


\begin{document}

%TODO don't need result, just use cur?

\begin{center}
{\huge{Result dictionaries with SQLite - a note on Lecture 7}}
\noindent\makebox[\linewidth]{\rule{\paperwidth}{0.4pt}}
\end{center}

\vskip 0.5cm

Last week we looked at creating and executing SQL queries on our SQLite databases
from Python programs. At the very end of the lecture you were introduced to an
example similar to snippet \ref{list:tup} below.

\begin{listing}[H]
    \caption[]{Executing a simple query on an SQLite database from Python}
    \label{list:tup}
    \begin{minted}[mathescape,
                %linenos,
                numbersep=5pt,
                gobble=8,
                frame=lines,
                framesep=2mm]{python}

        import sqlite3

        # Establish database connection and cursor
        con = sqlite3.connect("test.db")
        cur = con.cursor()

        # Execute a query
        result = cur.execute("SELECT postcode, streetname, house_number FROM address")

        # Do something with the resulting row tuples
        for row in result:
            # Output each address as "<house_number> <streetname>, <postcode>"
            print(row[2] + " " row[1] + ", " + row[0])

        # Close the database connection
        con.close()

    \end{minted}
\end{listing}

Here, we establish a connection to our \texttt{test.db} database using the
\texttt{sqlite3} package\footnote{{\href{https://docs.python.org/2/library/sqlite3.html}{https://docs.python.org/2/library/sqlite3.html}}}
and get a cursor from the connection to allow us to make queries. We then use
the cursor to execute an SQL query to fetch the house number, street name
and postcode for each property in the \texttt{address} table. To output the
addresses to the terminal, we then use a for loop over the \texttt{result} cursor
to read the values for each returned \texttt{row} \texttt{tuple} in turn.

You will notice that the ordering of the fields in the query maps to the index
of the values in each \texttt{row} tuple. For example, the postcode field appears
first in our query and is accessed via \texttt{row[0]}.

\pagebreak

But what if you want to access fields by their name? Snippet \ref{list:row} shows
how to alter an attribute of the connection such that cursors do not return rows
as a \texttt{tuple}, but as a special built-in \texttt{sqlite.Row} type - which
amongst a few other things, allows access to fields returned in a row by name in
a dictionary-like fashion.

\begin{listing}[H]
    \caption[]{Changing an attribute of \texttt{con} to return resulting rows as an \texttt{sqlite.Row} rather than a \texttt{tuple}}
    \label{list:row}
    \begin{minted}[mathescape,
                %linenos,
                numbersep=5pt,
                gobble=8,
                frame=lines,
                framesep=2mm]{python}

        import sqlite3

        # Establish database connection
        con = sqlite3.connect("test.db")

        # Set the row_factory attribute of the database connection to sqlite3.Row
        # This causes result rows to be returned as sqlite.Row rather than tuples
        con.row_factory = sqlite3.Row

        # Now get the cursor
        #   If we had set the row_factory after the cursor, the new factory would
        #   not have been applied to the cursor and we'd still get tuples back...
        cur = con.cursor()

        # Execute a query
        result = cur.execute("SELECT postcode, streetname, house_number FROM address")

        # Do something with the resulting rows
        for row in result:
            # We can now access field values by their name
            # Output each address as "<house_number> <streetname>, <postcode>"
            print(row["house_number"] + " " + row["streetname"] + ", " + row["postcode"])

        # Close the database connection
        con.close()

    \end{minted}
\end{listing}

%TODO package dot syntax?
The \texttt{row\_factory} attribute of the database connection is a function
responsible for the format in which cursors return rows from a query. As we've seen
in snippet \ref{list:tup}, by default the \texttt{row\_factory}
manufactures a \texttt{tuple} for each row.
However, as mentioned the \texttt{sqlite3} package includes an
optimised \texttt{Row} type\footnote{Hopefully you will remember from your experiments
with importing and using functions from your weather package, that the \texttt{Row} type
of the \texttt{sqlite3} package is accessed via \texttt{sqlite3.Row} - assuming \texttt{sqlite3}
has been imported of course.} which can
be assigned to the connection's \texttt{row\_factory} as demonstrated.

Note that this must be done before you request the cursor from the connection,
else the cursor will have the default \texttt{row\_factory} given to it by the connection.
Finally, as you can see, we are now able to access fields with dictionary-like syntax.

Don't forget to close your connections!

\end{document}
