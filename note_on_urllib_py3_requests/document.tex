\documentclass[a4paper]{article}

% Margins to 2cm
\usepackage[left=2cm,right=2cm,top=3cm,bottom=2cm]{geometry}

% No identation on paragraphs
\usepackage{parskip}

% Allow for code samples and change captions for "Listings" to "Code Snippet"
\usepackage{minted}
\usepackage{listings}
\renewcommand\listingscaption{Code Snippet}

% URLS
\usepackage{hyperref}

% Prevent hypenation (not the nicest way to do this)
\hyphenpenalty 1000

% Asterism
% http://tex.stackexchange.com/questions/194898/how-to-use-an-asterism-to-separate-sections
\newcommand{\asterism}{\bigskip\par\centerline{*\,*\,*}\medskip\par}%


\begin{document}

\begin{center}
    {\huge{A note on \texttt{urllib} and Python 3}}\\
    {\LARGE{and a short introduction to \texttt{requests}}}
\noindent\makebox[\linewidth]{\rule{\paperwidth}{0.4pt}}
\end{center}

\vskip 0.5cm

Towards the end of our penultimate lecture, a question was raised by someone in
the class about an error they had been encountered when trying to run one of the
XML examples (see snippet \ref{list:ex1} below) with Python 3.

\begin{listing}[H]
    \caption[]{Example API request as introduced at the end of Lecture 9}
    \label{list:ex1}
    \begin{minted}[mathescape,
                %linenos,
                numbersep=5pt,
                gobble=8,
                frame=lines,
                framesep=2mm]{python}

        from urllib2 import Request, urlopen, URLError
        import xml.etree.ElementTree as ET

        request = Request("http://users.aber.ac.uk/mfc1/csm0120/api.php?"
                          "uuid0=0BA846D3-CBA2-4D81-AD4C-00004868EA28")
        try:
            tree = ET.parse(urlopen(request))
            # Do something...
        except URLError, e:
            print(e)

    \end{minted}
\end{listing}

The example is fairly simple; import the necessary functions from the \texttt{urllib2}
and \texttt{xml} packages and create a \texttt{Request} to the desired API endpoint. The code
then attempts to open the \texttt{Request}'s URL with \texttt{urlopen} and
catches any \texttt{URLError}s that may occur in the process. Snippet \ref{list:err1}
demonstrates what happens when running this code in an interpreter using Python 3.

\begin{listing}[H]
    \caption[]{Error encountered running the example with Python 3}
    \label{list:err1}
    \begin{minted}[mathescape,
                %linenos,
                numbersep=5pt,
                gobble=8,
                frame=lines,
                framesep=2mm]{python}

        >>> from urllib2 import Request, urlopen, URLError
        Traceback (most recent call last):
        File "<stdin>", line 1, in <module>
        ImportError: No module named "urllib2"

    \end{minted}
\end{listing}

The \texttt{urllib2} documentation\footnote{{\href{https://docs.python.org/2/library/urllib2.html}{https://docs.python.org/2/library/urllib2.html}}} informs us that in Python 3, the module has been split into a
pair of smaller modules; \texttt{urllib.request} and \texttt{urllib.error}, and so
the \texttt{urllib2} module doesn't exist in the standard library anymore.
Snippet \ref{list:python3} demonstrates how to import these packages and ``fix"
the original example.

\begin{listing}[H]
    \caption[]{Updating the API example for Python 3}
    \label{list:python3}
    \begin{minted}[mathescape,
                linenos,
                numbersep=5pt,
                gobble=8,
                frame=lines,
                framesep=2mm]{python}

        from urllib.request import Request, urlopen
        from urllib.error import URLError
        import xml.etree.ElementTree as ET

        request = Request("http://users.aber.ac.uk/mfc1/csm0120/api.php?"
                          "uuid0=0BA846D3-CBA2-4D81-AD4C-00004868EA28")
        try:
            tree = ET.parse(urlopen(request))
            # Do something...
        except URLError as e:
            print(e)

    \end{minted}
\end{listing}

\pagebreak

A subtle change you may not have noticed appears in the error handling on line
10 of snippet \ref{list:python3}. A `Python Enhancement Proposal' submitted as part of
the Python 3 specification\footnote{PEP 3110 - Catching Exceptions in Python 3000 --- {\href{https://www.python.org/dev/peps/pep-3110/}{https://www.python.org/dev/peps/pep-3110/}}}  cleaned up a potential ambiguity
when using the \texttt{except} keyword and now requires use of the \texttt{as}
keyword instead of the comma we have been using. Thankfully this syntax is valid in
Python 2, too.

But this is a bit frustrating, what if we want a script to work for users without
having to worry what version of Python they are using? Whilst it is possible to
obtain the Python version number from the \texttt{sys} package as shown in
snippet \ref{list:get_version}, it may not always be correct (depending on how
the different versions of Python are actually installed).

\begin{listing}[H]
    \caption[]{Fetching Python's version number using the \texttt{sys} package}
    \label{list:get_version}
    \begin{minted}[mathescape,
                %linenos,
                numbersep=5pt,
                gobble=8,
                frame=lines,
                framesep=2mm]{python}

        sam@fi ~> python
        >>> import sys
        >>> sys.version
        "2.7.5 (default, Jun 26 2014, 11:55:39) \n[GCC 4.8.2 20131212 (Red Hat 4.8.2-7)]"
        >>> sys.version_info
        sys.version_info(major=2, minor=7, micro=5, releaselevel="final", serial=0)
        >>> sys.version_info[:2]
        (2, 7)

        sam@fi ~> python3
        >>> import sys
        >>> sys.version
        "3.3.2 (default, Jun 30 2014, 12:45:18) \n[GCC 4.8.2 20131212 (Red Hat 4.8.2-7)]"
        >>> sys.version_info
        sys.version_info(major=3, minor=3, micro=2, releaselevel="final", serial=0)
        >>> sys.version_info[:2]
        (3, 3)

    \end{minted}
\end{listing}

However, all is not lost. There is an even easier way that doesn't even involve
inspecting the contents of the \texttt{version\_info[:2]} tuple. You should now
be familiar with \texttt{Exceptions} in Python and how we can use the \texttt{try}
and \texttt{except} keywords to `catch' errors our code may encounter.
Snippet \ref{list:import_except} shows how we can use the fact that Python raises
an \texttt{ImportError} when a problem is encountered with importing modules to
our advantage.

\begin{listing}[H]
    \caption[]{Taking advantage of \texttt{ImportError} to support Python 2 and 3}
    \label{list:import_except}
    \begin{minted}[mathescape,
                %linenos,
                numbersep=5pt,
                gobble=8,
                frame=lines,
                framesep=2mm]{python}

        try:
            # Try and import using Python 2 compatible package
            from urllib2 import Request, urlopen, URLError
        except ImportError:
            # Python 3 will raise an ImportError as the urllib2 package does not exist
            # and so the imports we place in this block will be run - only by Python 3!
            from urllib.request import Request, urlopen
            from urllib.error import URLError

    \end{minted}
\end{listing}

Simple! When Python 3 raises the \texttt{ImportError} as we saw in snippet \ref{list:err1},
we can catch it and run the necessary imports for the program to work. Python 2
doesn't raise the \texttt{ImportError} and so skips over the additional imports
inside the \texttt{except} block\footnote{There is some loss of generality in this example.
Python 2 and 3 both encompass several different `minor' versions and some packages
actually differ between these minor versions. In fact the \texttt{xml} library we
have been using has been known as four different packages in its lifetime! We don't
need to worry about this as it's been the same since Python 2.5, but if you needed
to support very old versions of Python (a \textbf{very rare} occurrence), you'd need to catch all three possible \texttt{ImportError}s! See: Stack Overflow: Best way to import version-specific python modules ---
{\href{http://stackoverflow.com/questions/342437/best-way-to-import-version-specific-python-modules}{http://stackoverflow.com/questions/342437/best-way-to-import-version-specific-python-modules}}}.

\pagebreak
\asterism

As an aside, at the end of our final lecture, I mentioned my personal preference
for the \texttt{Requests}\footnote{{\href{http://docs.python-requests.org/}{http://docs.python-requests.org/}}} package. Self styled as ``HTTP for Humans" its authors
believe it is superior and simpler to use than the older \texttt{urllib2} package.
The package is compatible with all version of Python that you have met on this course
and according to the Anaconda package documentation\footnote{{\href{http://docs.continuum.io/anaconda/pkg-docs.html}{http://docs.continuum.io/anaconda/pkg-docs.html}}},
should be installed on your system already\footnote{I've confirmed this for Python 2.7
but not for Python 3.3. Instructions on installing further packages with \texttt{conda}
can be found at {\href{http://docs.continuum.io/anaconda/faq.html\#install-packages}{http://docs.continuum.io/anaconda/faq.html\#install-packages}}}.

Diving straight to an example, snippet \ref{list:requests_101} presents our earlier
example with \texttt{requests} in place of \texttt{urllib2}:

\begin{listing}[H]
    \caption[]{Transplanting \texttt{requests} in to our earlier example}
    \label{list:requests_101}
    \begin{minted}[mathescape,
                linenos,
                numbersep=5pt,
                gobble=8,
                frame=lines,
                framesep=2mm]{python}

        import requests
        import xml.etree.ElementTree as ET

        try:
            # Make the request
            response = requests.get("http://users.aber.ac.uk/mfc1/csm0120/api.php?"
                                    "uuid0=0BA846D3-CBA2-4D81-AD4C-00004868EA28", stream=True)
        except requests.exceptions.RequestException as e:
            # Catch any possible exceptions
            print(e)
        else:
            tree = ET.parse(response.raw)
            # Do something...

    \end{minted}
\end{listing}

Here the \texttt{Request} and \texttt{urlopen} steps of snippet \ref{list:ex1} are
effectively rolled in to one function call to \texttt{requests.get} whose result
is saved in the \texttt{response} variable. Note the
\texttt{stream=True} keyword argument (line 7) which forces the get function to
return a file-like response from the API which is what the \texttt{ElementTree}
parser expects on line 12. We try to catch any possible \texttt{requests.exceptions.RequestException}s
that may occur, else we move on to create the tree as before. Note also we pass
\texttt{response.raw} to the parser, for the same reason as before\footnote{We don't need
to go in to much detail here but if we did not set \texttt{stream=True} and tried
to just pass the plain XML response string to the parser, \texttt{parse} would assume
this string was a file name and fail to build a tree. We could use \texttt{fromString} instead,
but as we saw in the final lecture, that can lead to some troubles of its own and
we're trying to reduce the number of changes needed to use this different package!}.

It might seem like we've overcomplicated the example somewhat with this stream
and raw stuff that I'm trying to circumvent a lengthy explanation of, but the benefits of
the \texttt{requests} package become more obvious when we try to improve
on the initial example like we have done in snippet \ref{list:requests_w_payload}:

\asterism
\pagebreak

\begin{listing}[H]
    \caption[]{A somewhat improved example using some other features of \texttt{requests}}
    \label{list:requests_w_payload}
    \begin{minted}[mathescape,
                linenos,
                numbersep=5pt,
                gobble=8,
                frame=lines,
                framesep=2mm]{python}

        import requests
        import xml.etree.ElementTree as ET

        # Populate a "payload" dictionary to send to the API
        payload = {"uuid0": "0BA846D3-CBA2-4D81-AD4C-00004868EA28",
                "county0": "GREATER MANCHESTER"}

        try:
            # Make the request
            response = requests.get("http://users.aber.ac.uk/mfc1/csm0120/api.php",
                                        params=payload, stream=True)
        except requests.exceptions.RequestException as e:
            # Catch any possible exceptions
            print(e)
        else:
            # Check whether the request was actually successful
            if response.status_code == 200:
                tree = ET.parse(response.raw)
                # Do something...
            else:
                # If the status is not HTTP 200 (OK), raise the status as an HTTPError
                print(response.raise_for_status())


    \end{minted}
\end{listing}

Firstly we populate a dictionary with the key-value pairs that we want to use to
query the API (see Assignment 2 for further information). Here we ask for some
information on a particular house sale and the geographical co-ordinates of a
county that happens to contain a sneaky space character.

We've added a second keyword argument on line 11 to the \texttt{requests.get} function;
\texttt{params=payload}. This \texttt{params} argument accepts a dictionary which
is used to build the URL of the API request automatically, including escaping
characters such as spaces where necessary\footnote{You may wonder why the space has
been encoded to \texttt{+} instead of the \texttt{\%20} we were shown in class. The plus
symbol is actually a valid replacement for the space character when being used in
a query string of an URL, you'll notice other non-alphanumeric characters are correctly
encoded to their \%XX counterparts} (see snippet \ref{list:requests_w_params}).
Note we no longer need the ? at the end of the URL on line 10 either, it's taken care of.

Whilst not necessary due to the nature of this assignment and the API itself,
in a real world application we might also want to check
the status code of the response from the API. You've most likely come across `404' errors
on the internet for pictures of cats that can no longer be found, such errors are not raised
by \texttt{RequestException} as a `404' is still a valid response. Many APIs will return a status
code of `400' for ``Bad Request"\footnote{You can
find a full list on Wikipedia: List of HTTP status codes --- {\href{https://en.wikipedia.org/wiki/List\_of\_HTTP\_status\_codes}{https://en.wikipedia.org/wiki/List\_of\_HTTP\_status\_codes}}}
for incorrect or missing parameters in your query.

As shown on line 17, we can test whether the \texttt{status\_code} attribute of our
\texttt{response} is equal to the number 200, which represents the status ``OK".
If it is, we can get on with building the tree as before, otherwise we use the
response's special \texttt{raise\_for\_status} function to elevate the HTTP status
code to a proper exception; \texttt{HTTPError} that will stop the program and print
an error. HTTP for Humans indeed!
\begin{listing}[H]
    \caption[]{Given a dictionary \texttt{params}, \texttt{requests} can build your API endpoint URL for you}
    \label{list:requests_w_params}
    \begin{minted}[mathescape,
                numbersep=5pt,
                gobble=8,
                frame=lines,
                framesep=2mm]{python}
        >>> import requests
        >>> response = requests.get("http://users.aber.ac.uk/mfc1/csm0120/api.php", params={
        >>>     "uuid0": "0BA846D3-CBA2-4D81-AD4C-00004868EA28",
        >>>     "county0": "GREATER MANCHESTER", "county1": ">^_^< i am a cat"})
        >>> response.url
        u"http://users.aber.ac.uk/mfc1/csm0120/api.php?county1=%3E%5E_%5E%3C+i+am+a+cat&"\
        "county0=GREATER+MANCHESTER&uuid0=0BA846D3-CBA2-4D81-AD4C-00004868EA28"

    \end{minted}
\end{listing}

\end{document}
